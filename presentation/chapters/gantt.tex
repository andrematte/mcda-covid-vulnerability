\section{Conclusão}

\begin{frame}{Conclusões}

\begin{itemize}
    \item Identificação das capitais brasileiras mais vulneráveis às consequências da COVID-19;
    \item Baseado em critérios sociais, econômicos, demográficos e epidemiológios;
	\item Ranking de vulnerabilidade coerente com os resultados esperados (publicações, notícias);
	\item Capitais das regiões Norte e Nordeste são mais vulneráveis em relação as demais, com destaque para Manaus, AM;
	\item Direcionamento de medidas de controle e contenção da doença prioritariamente para regiões vulneráveis.
\end{itemize}
	
\end{frame}

\begin{frame}{Evolução do Trabalho}

\noindent\resizebox{\textwidth}{!}{
\begin{ganttchart}[
    hgrid,
    vgrid = {*{6}{draw=none}, dotted},
    x unit=0.18cm,
    time slot format=little-endian,
    time slot unit=day,
    calendar week text = {\currentweek{}},
    title label font=\bfseries\footnotesize,
    title height=1,
    inline,
    %bar height = .7, 
    %bar top shift = -0.01, 
    ]{22/03/2021}{27/06/2021}
    
    \gantttitlecalendar{month=shortname, week}\\

	\ganttbar{Proposta}{25-03-2021}{15-04-2021}\\
	\ganttbar{Metodologia}{5-04-2021}{25-04-2021}\\
	\ganttbar{Avaliação dos Critérios}{25-04-2021}{16-05-2021}\\
	\ganttbar{Implementação}{02-05-2021}{30-05-2021}\\
	\ganttbar{Validação}{16-05-2021}{30-05-2021}\\
	\ganttbar{Versão Final}{01-06-2021}{22-06-2021}
\end{ganttchart}
}


\end{frame} 

%----------------------------------------------------

\begin{frame}{Evolução do Trabalho}

\textbf{Dificuldades Encontradas:}
\begin{itemize}
	\item Definição dos critérios
	\item Método de seleção de pesos
\end{itemize}

\bigskip
\textbf{Divulgação de Resultados:}
\begin{itemize}
	\item Código e explicações disponibilizados em: github.com/andrematte/mcda-covid-vulnerability
	\item Artigo científico parcialmente escrito
\end{itemize}


	
\end{frame}
